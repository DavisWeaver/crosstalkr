\documentclass{article}

\usepackage{arxiv}

\usepackage[utf8]{inputenc} % allow utf-8 input
\usepackage[T1]{fontenc}    % use 8-bit T1 fonts
\usepackage{lmodern}        % https://github.com/rstudio/rticles/issues/343
\usepackage{hyperref}       % hyperlinks
\usepackage{url}            % simple URL typesetting
\usepackage{booktabs}       % professional-quality tables
\usepackage{amsfonts}       % blackboard math symbols
\usepackage{nicefrac}       % compact symbols for 1/2, etc.
\usepackage{microtype}      % microtypography
\usepackage{graphicx}

\title{crosstalkr: an R package for the identification of related nodes
in biological networks}

\author{
    Davis T. Weaver
   \\
    1, 2 \\
   \\
  \texttt{} \\
   \And
    Jacob Scott
   \\
    1,2 \\
   \\
  \texttt{} \\
  }

% Pandoc syntax highlighting
\usepackage{color}
\usepackage{fancyvrb}
\newcommand{\VerbBar}{|}
\newcommand{\VERB}{\Verb[commandchars=\\\{\}]}
\DefineVerbatimEnvironment{Highlighting}{Verbatim}{commandchars=\\\{\}}
% Add ',fontsize=\small' for more characters per line
\usepackage{framed}
\definecolor{shadecolor}{RGB}{248,248,248}
\newenvironment{Shaded}{\begin{snugshade}}{\end{snugshade}}
\newcommand{\AlertTok}[1]{\textcolor[rgb]{0.94,0.16,0.16}{#1}}
\newcommand{\AnnotationTok}[1]{\textcolor[rgb]{0.56,0.35,0.01}{\textbf{\textit{#1}}}}
\newcommand{\AttributeTok}[1]{\textcolor[rgb]{0.77,0.63,0.00}{#1}}
\newcommand{\BaseNTok}[1]{\textcolor[rgb]{0.00,0.00,0.81}{#1}}
\newcommand{\BuiltInTok}[1]{#1}
\newcommand{\CharTok}[1]{\textcolor[rgb]{0.31,0.60,0.02}{#1}}
\newcommand{\CommentTok}[1]{\textcolor[rgb]{0.56,0.35,0.01}{\textit{#1}}}
\newcommand{\CommentVarTok}[1]{\textcolor[rgb]{0.56,0.35,0.01}{\textbf{\textit{#1}}}}
\newcommand{\ConstantTok}[1]{\textcolor[rgb]{0.00,0.00,0.00}{#1}}
\newcommand{\ControlFlowTok}[1]{\textcolor[rgb]{0.13,0.29,0.53}{\textbf{#1}}}
\newcommand{\DataTypeTok}[1]{\textcolor[rgb]{0.13,0.29,0.53}{#1}}
\newcommand{\DecValTok}[1]{\textcolor[rgb]{0.00,0.00,0.81}{#1}}
\newcommand{\DocumentationTok}[1]{\textcolor[rgb]{0.56,0.35,0.01}{\textbf{\textit{#1}}}}
\newcommand{\ErrorTok}[1]{\textcolor[rgb]{0.64,0.00,0.00}{\textbf{#1}}}
\newcommand{\ExtensionTok}[1]{#1}
\newcommand{\FloatTok}[1]{\textcolor[rgb]{0.00,0.00,0.81}{#1}}
\newcommand{\FunctionTok}[1]{\textcolor[rgb]{0.00,0.00,0.00}{#1}}
\newcommand{\ImportTok}[1]{#1}
\newcommand{\InformationTok}[1]{\textcolor[rgb]{0.56,0.35,0.01}{\textbf{\textit{#1}}}}
\newcommand{\KeywordTok}[1]{\textcolor[rgb]{0.13,0.29,0.53}{\textbf{#1}}}
\newcommand{\NormalTok}[1]{#1}
\newcommand{\OperatorTok}[1]{\textcolor[rgb]{0.81,0.36,0.00}{\textbf{#1}}}
\newcommand{\OtherTok}[1]{\textcolor[rgb]{0.56,0.35,0.01}{#1}}
\newcommand{\PreprocessorTok}[1]{\textcolor[rgb]{0.56,0.35,0.01}{\textit{#1}}}
\newcommand{\RegionMarkerTok}[1]{#1}
\newcommand{\SpecialCharTok}[1]{\textcolor[rgb]{0.00,0.00,0.00}{#1}}
\newcommand{\SpecialStringTok}[1]{\textcolor[rgb]{0.31,0.60,0.02}{#1}}
\newcommand{\StringTok}[1]{\textcolor[rgb]{0.31,0.60,0.02}{#1}}
\newcommand{\VariableTok}[1]{\textcolor[rgb]{0.00,0.00,0.00}{#1}}
\newcommand{\VerbatimStringTok}[1]{\textcolor[rgb]{0.31,0.60,0.02}{#1}}
\newcommand{\WarningTok}[1]{\textcolor[rgb]{0.56,0.35,0.01}{\textbf{\textit{#1}}}}

% tightlist command for lists without linebreak
\providecommand{\tightlist}{%
  \setlength{\itemsep}{0pt}\setlength{\parskip}{0pt}}

% From pandoc table feature
\usepackage{longtable,booktabs,array}
\usepackage{calc} % for calculating minipage widths
% Correct order of tables after \paragraph or \subparagraph
\usepackage{etoolbox}
\makeatletter
\patchcmd\longtable{\par}{\if@noskipsec\mbox{}\fi\par}{}{}
\makeatother
% Allow footnotes in longtable head/foot
\IfFileExists{footnotehyper.sty}{\usepackage{footnotehyper}}{\usepackage{footnote}}
\makesavenoteenv{longtable}

% Pandoc citation processing
\newlength{\cslhangindent}
\setlength{\cslhangindent}{1.5em}
\newlength{\csllabelwidth}
\setlength{\csllabelwidth}{3em}
\newlength{\cslentryspacingunit} % times entry-spacing
\setlength{\cslentryspacingunit}{\parskip}
% for Pandoc 2.8 to 2.10.1
\newenvironment{cslreferences}%
  {}%
  {\par}
% For Pandoc 2.11+
\newenvironment{CSLReferences}[2] % #1 hanging-ident, #2 entry spacing
 {% don't indent paragraphs
  \setlength{\parindent}{0pt}
  % turn on hanging indent if param 1 is 1
  \ifodd #1
  \let\oldpar\par
  \def\par{\hangindent=\cslhangindent\oldpar}
  \fi
  % set entry spacing
  \setlength{\parskip}{#2\cslentryspacingunit}
 }%
 {}
\usepackage{calc}
\newcommand{\CSLBlock}[1]{#1\hfill\break}
\newcommand{\CSLLeftMargin}[1]{\parbox[t]{\csllabelwidth}{#1}}
\newcommand{\CSLRightInline}[1]{\parbox[t]{\linewidth - \csllabelwidth}{#1}\break}
\newcommand{\CSLIndent}[1]{\hspace{\cslhangindent}#1}

\begin{document}
\maketitle


\begin{abstract}

\end{abstract}


\hypertarget{summary}{%
\section{Summary}\label{summary}}

Crosstalkr provides a unified toolkit for drug target and disease
subnetwork identification. First, Crosstalkr enables users to download
and leverage high-quality protein-protein interaction networks from
online repositories. Users can then filter these large networks into
manageable subnetworks using a variety of methods.\\
For example, network filtration can be done using random walks with
restarts, starting at the user-provided seed proteins. Affinity scores
from a given random walk with restarts are compared to a bootstrapped
null distribution to assess statistical significance. Random walks are
implemented using sparse matrix multiplication to facilitate fast
execution. Next, users can perform in-silico repression experiments to
assess the relative importance of nodes in their network. At this step,
users can supply protein or gene expression data to make node rankings
more meaningful. The default behavior evaluates the human interactome.
However, users can evaluate more than 1000 non-human protein-protein
interaction networks thanks to integration with StringDB. It is a free,
open-source R package designed to allow users to integrate functional
analysis using the protein-protein interaction network into existing
bioinformatic pipelines.

\hypertarget{statement-of-need}{%
\section{Statement of Need}\label{statement-of-need}}

In the last few decades, interest in graph-based analysis of biological
networks has grown substantially. Protein-protein interaction networks
are one of the most common biological networks, and represent the
molecular relationships between every known protein and every other
known protein.

Researchers have applied graph search and graph clustering algorithms to
biological networks in an effort to derive disease-specific subgraphs
(Nibbe, Koyutürk, and Chance 2010; Chitra, Park, and Raphael 2022;
Pfeifer et al. 2022) or identify potential drug targets (Weaver et al.
2021; Martínez et al. 2015). One of the most well-studied algorithms in
this context is the random walk with restarts. Random walks with
restarts (RWR) have been used and adapted across disciplines and
industries for applications ranging from internet search engines to drug
target identification. (Tong, Faloutsos, and Pan 2006; Bianchini, Gori,
and Scarselli 2005; Navarro et al. 2017)

There is a growing suite of tools available in R for analyzing
graph-structured data (details 2022; Gatto and Christoforou 2014),
including a few R packages that implement RWR (Fang and Gough 2014;
Valentini 2022). These include the RANKS package, which provides tools
for performing many graph-based node scoring algorithms.(Valentini
(2022)). These tools require some understanding of graph data structures
and ask the user to find, download, and manipulate the relevant
biological networks into adjacency matrices or igraph objects.
Crosstalkr compromises some of the flexibility of RANKS to provide an
optimized, streamlined interface to allow users to integrate
interactomic analyses into their workflow. While users can interact
directly with crosstalkr to perform RWR on any graph, the package is
optimized to facilitate one-line implementation of an algorithm designed
to identify functional subgraphs of protein-protein interaction networks
(PPI).

\hypertarget{design-and-data-sources}{%
\section{Design and Data Sources}\label{design-and-data-sources}}

\hypertarget{compute_crosstalk}{%
\subsection{compute\_crosstalk}\label{compute_crosstalk}}

The main entrypoint for most users will be the compute\_crosstalk
function. If users plan to search a supported protein protein
interaction network, they are only required to provide a vector of seed
proteins. In this situation, compute\_crosstalk will:

\begin{enumerate}
\def\labelenumi{\arabic{enumi}.}
\tightlist
\item
  Download the requested PPI (or load it from the provided cache)
\item
  Process the requested PPI into a sparse adjacency matrix.
\item
  Perform a random walk with restart using the user provided seeds to
  generate affinity scores for every protein in the PPI.
\item
  Perform many random walks with restarts from n random seeds with a
  matching degree distribution to generate a null distribution of
  affinity score.
\item
  Compare the affinity scores to the null distribution to compute an
  adjusted p-value (using the method specified in p\_adjust)
\item
  Remove proteins with an adjusted p-value \textless{}
  significance\_level
\end{enumerate}

Users can make use of caching to store processed PPIs and speed up
future analyses substantially. Users can also make use of parallel
computing by setting the ncores parameter \textgreater{} 1 A sample
workflow demonstrating the ease of use is provided below. Here, we
attempt to determine proteins that are functionally related to EGFR,
KRAS, PI3K, and STAT3; proteins that are involved in growth signaling in
cancer cells.

\begin{Shaded}
\begin{Highlighting}[]
\NormalTok{df }\OtherTok{\textless{}{-}} \FunctionTok{compute\_crosstalk}\NormalTok{(}\AttributeTok{seed\_proteins =} \FunctionTok{c}\NormalTok{(}\StringTok{"EGFR"}\NormalTok{, }\StringTok{"KRAS"}\NormalTok{,}\StringTok{"STAT3"}\NormalTok{), }
                        \AttributeTok{cache =} \StringTok{"./data"}\NormalTok{, }\AttributeTok{seed\_name =} \StringTok{"joss\_ex"}\NormalTok{, }\AttributeTok{n =} \DecValTok{1000}\NormalTok{, }
                        \AttributeTok{significance\_level =} \FloatTok{0.99}\NormalTok{)}
\NormalTok{df }\SpecialCharTok{\%\textgreater{}\%}
  \FunctionTok{select}\NormalTok{(}\SpecialCharTok{{-}}\FunctionTok{c}\NormalTok{(Z,mean\_p, var\_p, nobs)) }\SpecialCharTok{\%\textgreater{}\%}
  \FunctionTok{slice\_max}\NormalTok{(}\AttributeTok{order\_by =}\NormalTok{ affinity\_score, }\AttributeTok{n =} \DecValTok{5}\NormalTok{) }\SpecialCharTok{\%\textgreater{}\%} 
\NormalTok{  knitr}\SpecialCharTok{::}\FunctionTok{kable}\NormalTok{(}\AttributeTok{digits =} \DecValTok{4}\NormalTok{)}
\end{Highlighting}
\end{Shaded}

\begin{longtable}[]{@{}llrrr@{}}
\toprule
node & seed & affinity\_score & p\_value & adj\_p\_value \\
\midrule
\endhead
STAT3 & yes & 0.2002 & 0 & 0e+00 \\
EGFR & yes & 0.2001 & 0 & 0e+00 \\
KRAS & yes & 0.2001 & 0 & 0e+00 \\
C2orf72 & no & 0.0041 & 0 & 1e-04 \\
CCDC87 & no & 0.0029 & 0 & 0e+00 \\
\bottomrule
\end{longtable}

We also provide a convenience function to quickly plot the returned
subgraph. Users can specify \texttt{prop\_keep} to improve readability
by only plotting the top x\% of identified proteins, ranked by affinity
score.

\begin{Shaded}
\begin{Highlighting}[]
\NormalTok{g }\OtherTok{\textless{}{-}} \FunctionTok{prep\_stringdb}\NormalTok{(}\AttributeTok{cache =} \StringTok{"./data"}\NormalTok{)}
\FunctionTok{png}\NormalTok{(}\AttributeTok{filename =} \StringTok{\textquotesingle{}ct\_plot\_ex.png\textquotesingle{}}\NormalTok{)}
\NormalTok{crosstalkr}\SpecialCharTok{::}\FunctionTok{plot\_ct}\NormalTok{(df, }\AttributeTok{g=}\NormalTok{g, }\AttributeTok{prop\_keep =} \FloatTok{0.4}\NormalTok{, }\AttributeTok{label\_prop =} \FloatTok{0.2}\NormalTok{)}
\FunctionTok{dev.off}\NormalTok{()}
\end{Highlighting}
\end{Shaded}

\begin{verbatim}
## pdf 
##   2
\end{verbatim}

\hypertarget{other-features}{%
\subsection{Other Features}\label{other-features}}

In pursuit of a one-line interactomic analysis pipeline, we developed
several convencience functions that users will likely find useful in
other analyses. For example, the human protein-protein interaction
network crosstalkr is able to detect and convert between entrez ids,
uniprot names, and ensemble ids. Users can make use of the
as\_gene\_symbol function to convert any common representation of gene
identity into human-readable gene names. In addition, users can make use
of single-line functions that download and clean PPIs from either
StringDB or Biogrid. Crosstalkr also ships with a highly optimized
implementation of the random-walk with restarts algorithm (sparseRWR),
which users can apply to any graph-structured data.

\hypertarget{data-sources}{%
\subsection{Data sources}\label{data-sources}}

Users can leverage two high quality PPIs through crosstalkr; StringDB
and Biogrid (Oughtred et al. 2021; Szklarczyk et al. 2021). Users can
run their analysis using either of these resources individually or they
can take the union or intersection of these networks. While Biogrid only
supports the human PPI, StringDB provides high quality PPIs for more
than 1500 species (Szklarczyk et al. 2021). Crosstalkr provides a
user-friendly interface for all of these species.

\hypertarget{acknowledgements}{%
\section{Acknowledgements}\label{acknowledgements}}

We acknowledge contributions from Mark Chance and Mehmet Koyuturk. We
would also like to acknowledge the dependencies that enable crosstalkr
(Wickham, François, et al. 2022; details 2022; magrittr) et al. 2022;
Hester et al. 2022; Bates et al. 2022; Wickham, Hester, et al. 2022;
Wickham, Girlich, and RStudio 2022; Daniel, Ooi, et al. 2022; Daniel,
Corporation, et al. 2022; Rainer, Gatto, and Weichenberger 2019) and
this paper (Xie {[}aut et al. 2022; Wickham, Chang, et al. 2022)

\hypertarget{citations}{%
\section*{Citations}\label{citations}}
\addcontentsline{toc}{section}{Citations}

\hypertarget{refs}{}
\begin{CSLReferences}{1}{0}
\leavevmode\hypertarget{ref-bates_matrix_2022}{}%
Bates, Douglas, Martin Maechler, Mikael Jagan, Timothy A. Davis
(SuiteSparse and 'cs' C. libraries, notably CHOLMOD AMD, collaborators
listed in and dir(pattern="\^{}+txt\$", full.names=TRUE, et al. 2022.
{``Matrix: {Sparse} and {Dense} {Matrix} {Classes} and {Methods}.''}
\url{https://CRAN.R-project.org/package=Matrix}.

\leavevmode\hypertarget{ref-bianchini_inside_2005}{}%
Bianchini, Monica, Marco Gori, and Franco Scarselli. 2005. {``Inside
{PageRank}.''} \emph{ACM Transactions on Internet Technology} 5 (1):
92--128. \url{https://doi.org/10.1145/1052934.1052938}.

\leavevmode\hypertarget{ref-chitra_netmix2_2022}{}%
Chitra, Uthsav, Tae Yoon Park, and Benjamin J. Raphael. 2022.
{``{NetMix2}: {Unifying} {Network} {Propagation} and {Altered}
{Subnetworks}.''} In \emph{Research in {Computational} {Molecular}
{Biology}}, 193--208.
\url{https://doi.org/10.1007/978-3-031-04749-7_12}.

\leavevmode\hypertarget{ref-daniel_doparallel_2022}{}%
Daniel, Folashade, Microsoft Corporation, Steve Weston, and Dan
Tenenbaum. 2022. {``{doParallel}: {Foreach} {Parallel} {Adaptor} for the
'Parallel' {Package}.''}
\url{https://CRAN.R-project.org/package=doParallel}.

\leavevmode\hypertarget{ref-daniel_foreach_2022}{}%
Daniel, Folashade, Hong Ooi, Rich Calaway, Microsoft, and Steve Weston.
2022. {``Foreach: {Provides} {Foreach} {Looping} {Construct}.''}
\url{https://CRAN.R-project.org/package=foreach}.

\leavevmode\hypertarget{ref-details_igraph_2022}{}%
details, See AUTHORS file igraph author. 2022. {``Igraph: {Network}
{Analysis} and {Visualization}.''}
\url{https://CRAN.R-project.org/package=igraph}.

\leavevmode\hypertarget{ref-fang_dnet_2014}{}%
Fang, Hai, and Julian Gough. 2014. {``The `Dnet' Approach Promotes
Emerging Research on Cancer Patient Survival.''} \emph{Genome Medicine}
6 (8): 64. \url{https://doi.org/10.1186/s13073-014-0064-8}.

\leavevmode\hypertarget{ref-gatto_using_2014}{}%
Gatto, Laurent, and Andy Christoforou. 2014. {``Using {R} and
{Bioconductor} for Proteomics Data Analysis.''} \emph{Biochimica Et
Biophysica Acta (BBA) - Proteins and Proteomics} 1844 (1): 42--51.
\url{https://doi.org/10.1016/j.bbapap.2013.04.032}.

\leavevmode\hypertarget{ref-hester_withr_2022}{}%
Hester, Jim, Lionel Henry, Kirill Müller, Kevin Ushey, Hadley Wickham,
Winston Chang, Jennifer Bryan, Richard Cotton, and RStudio. 2022.
{``Withr: {Run} {Code} '{With}' {Temporarily} {Modified} {Global}
{State}.''} \url{https://CRAN.R-project.org/package=withr}.

\leavevmode\hypertarget{ref-magrittr_magrittr_2022}{}%
magrittr), Stefan Milton Bache (Original author and creator of, Hadley
Wickham, Lionel Henry, and RStudio. 2022. {``Magrittr: {A}
{Forward}-{Pipe} {Operator} for {R}.''}
\url{https://CRAN.R-project.org/package=magrittr}.

\leavevmode\hypertarget{ref-martinez_drugnet_2015}{}%
Martínez, Víctor, Carmen Navarro, Carlos Cano, Waldo Fajardo, and
Armando Blanco. 2015. {``{DrugNet}: {Network}-Based Drug--Disease
Prioritization by Integrating Heterogeneous Data.''} \emph{Artificial
Intelligence in Medicine} 63 (1): 41--49.
\url{https://doi.org/10.1016/j.artmed.2014.11.003}.

\leavevmode\hypertarget{ref-navarro_prophtools_2017}{}%
Navarro, Carmen, Victor Martínez, Armando Blanco, and Carlos Cano. 2017.
{``{ProphTools}: General Prioritization Tools for Heterogeneous
Biological Networks.''} \emph{GigaScience} 6 (12): 1--8.
\url{https://doi.org/10.1093/gigascience/gix111}.

\leavevmode\hypertarget{ref-nibbe_integrative_2010}{}%
Nibbe, Rod K., Mehmet Koyutürk, and Mark R. Chance. 2010. {``An
{Integrative} -Omics {Approach} to {Identify} {Functional}
{Sub}-{Networks} in {Human} {Colorectal} {Cancer}.''} \emph{PLOS
Computational Biology} 6 (1): e1000639.
\url{https://doi.org/10.1371/journal.pcbi.1000639}.

\leavevmode\hypertarget{ref-oughtred_biogrid_2021}{}%
Oughtred, Rose, Jennifer Rust, Christie Chang, Bobby‐Joe Breitkreutz,
Chris Stark, Andrew Willems, Lorrie Boucher, et al. 2021. {``The
{BioGRID} Database: {A} Comprehensive Biomedical Resource of Curated
Protein, Genetic, and Chemical Interactions.''} \emph{Protein Science :
A Publication of the Protein Society} 30 (1): 187--200.
\url{https://doi.org/10.1002/pro.3978}.

\leavevmode\hypertarget{ref-pfeifer_gnn-subnet_2022}{}%
Pfeifer, Bastian, Afan Secic, Anna Saranti, and Andreas Holzinger. 2022.
{``{GNN}-{SubNet}: Disease Subnetwork Detection with Explainable {Graph}
{Neural} {Networks}.''} \emph{bioRxiv}, 2022.01.12.475995.
\url{https://doi.org/10.1101/2022.01.12.475995}.

\leavevmode\hypertarget{ref-rainer_ensembldb_2019}{}%
Rainer, Johannes, Laurent Gatto, and Christian X Weichenberger. 2019.
{``Ensembldb: An {R} Package to Create and Use {Ensembl}-Based
Annotation Resources.''} \emph{Bioinformatics} 35 (17): 3151--53.
\url{https://doi.org/10.1093/bioinformatics/btz031}.

\leavevmode\hypertarget{ref-szklarczyk_string_2021}{}%
Szklarczyk, Damian, Annika L. Gable, Katerina C. Nastou, David Lyon,
Rebecca Kirsch, Sampo Pyysalo, Nadezhda T. Doncheva, et al. 2021. {``The
{STRING} Database in 2021: Customizable Protein-Protein Networks, and
Functional Characterization of User-Uploaded Gene/Measurement Sets.''}
\emph{Nucleic Acids Research} 49 (D1): D605--12.
\url{https://doi.org/10.1093/nar/gkaa1074}.

\leavevmode\hypertarget{ref-tong_fast_2006}{}%
Tong, Hanghang, Christos Faloutsos, and Jia-yu Pan. 2006. {``Fast
{Random} {Walk} with {Restart} and {Its} {Applications}.''} In
\emph{Sixth {International} {Conference} on {Data} {Mining}
({ICDM}'06)}, 613--22. \url{https://doi.org/10.1109/ICDM.2006.70}.

\leavevmode\hypertarget{ref-valentini_ranks_2022}{}%
Valentini, Giorgio. 2022. {``{RANKS}: {Ranking} of {Nodes} with
{Kernelized} {Score} {Functions}.''}
\url{https://CRAN.R-project.org/package=RANKS}.

\leavevmode\hypertarget{ref-weaver_network_2021}{}%
Weaver, Davis T., Kathleen I. Pishas, Drew Williamson, Jessica
Scarborough, Stephen L. Lessnick, Andrew Dhawan, and Jacob G. Scott.
2021. {``Network Potential Identifies Therapeutic {miRNA} Cocktails in
{Ewing} Sarcoma.''} \emph{PLOS Computational Biology} 17 (10): e1008755.
\url{https://doi.org/10.1371/journal.pcbi.1008755}.

\leavevmode\hypertarget{ref-wickham_ggplot2_2022}{}%
Wickham, Hadley, Winston Chang, Lionel Henry, Thomas Lin Pedersen,
Kohske Takahashi, Claus Wilke, Kara Woo, Hiroaki Yutani, Dewey
Dunnington, and RStudio. 2022. {``Ggplot2: {Create} {Elegant} {Data}
{Visualisations} {Using} the {Grammar} of {Graphics}.''}
\url{https://CRAN.R-project.org/package=ggplot2}.

\leavevmode\hypertarget{ref-wickham_dplyr_2022}{}%
Wickham, Hadley, Romain François, Lionel Henry, Kirill Müller, and
RStudio. 2022. {``Dplyr: {A} {Grammar} of {Data} {Manipulation}.''}
\url{https://CRAN.R-project.org/package=dplyr}.

\leavevmode\hypertarget{ref-wickham_tidyr_2022}{}%
Wickham, Hadley, Maximilian Girlich, and RStudio. 2022. {``Tidyr: {Tidy}
{Messy} {Data}.''} \url{https://CRAN.R-project.org/package=tidyr}.

\leavevmode\hypertarget{ref-wickham_readr_2022}{}%
Wickham, Hadley, Jim Hester, Romain Francois, Jennifer Bryan, Shelby
Bearrows, RStudio, https://github com/mandreyel/ (mio library), Jukka
Jylänki (grisu3 implementation), and Mikkel Jørgensen (grisu3
implementation). 2022. {``Readr: {Read} {Rectangular} {Text} {Data}.''}
\url{https://CRAN.R-project.org/package=readr}.

\leavevmode\hypertarget{ref-xie__aut_knitr_2022}{}%
Xie {[}aut, Yihui, cre, Abhraneel Sarma, Adam Vogt, Alastair Andrew,
Alex Zvoleff, Amar Al-Zubaidi, et al. 2022. {``Knitr: {A}
{General}-{Purpose} {Package} for {Dynamic} {Report} {Generation} in
{R}.''} \url{https://CRAN.R-project.org/package=knitr}.

\end{CSLReferences}

\bibliographystyle{unsrt}
\bibliography{paper.bib}


\end{document}
